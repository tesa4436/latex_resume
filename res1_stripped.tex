% LaTeX file for resume 
% This file uses the resume document class (res.cls)

\documentclass{res} 
%\usepackage{helvetica} % uses helvetica postscript font (download helvetica.sty)
%\usepackage{newcent}   % uses new century schoolbook postscript font 
\setlength{\textheight}{9.5in} % increase text height to fit on 1-page 

\begin{document} 

\name{Teodoras Šaulys\\[12pt]}     % the \\[12pt] adds a blank
				        % line after name      

\address{\texttt{github.com/tesa4436}}
                                  
\begin{resume}

\section{CONTACT INFORMATION}          
\begin{itemize}
	\item Email 1:			\texttt{hello@world.com}
	\item Email 2:			\texttt{lorem@ipsum.org}
	\item Phone number:		1337696969
	\item Code repositories:	\texttt{github.com/tesa4436}
\end{itemize}


\section{EDUCATION}          

\begin{itemize}
    \item Vilnius University, Faculty of Mathematics and Informatics\\        
    Bachelor of Software Engineering, 2017-2021 (ongoing)      
\end{itemize}
 
\section{EXPERIENCE AND PERSONAL PROJECTS}
   \vspace{-0.1in}	
   \begin{tabbing}
   \hspace{2.3in}\= \hspace{2.97in}\= \kill % set up two tab positions
    \textbf{tdwm} (TeD's Window Manager), a window manager for the X Window System\\
    \texttt{github.com/tesa4436/tdwm}\>     \>Language: C\\
   \end{tabbing}\vspace{-20pt}      % suppress blank line after tabbing
   Unlike traditional window managers in Unix systems, \textbf{tdwm} places windows in a tiling layout,
   utilising the binary tree data structure. The screen real estate is completely covered in windows at
   all times, which enhances productivity and reduces clutter on the screen. The code is C99 compliant.
   \begin{tabbing}
   \hspace{2.3in}\= \hspace{1.9in}\= \kill % set up two tab positions
    \bf Intel 8086 Disassembler\\
    \texttt{github.com/tesa4436/disasm} \> \>Language: Intel 8086 assembly\\
   \end{tabbing}\vspace{-20pt}
   This program disassembles Intel 8086 machine code into assembly language mnemonics.
   The code is printed to the standard output.
   \begin{tabbing}
   \hspace{2.3in}\= \hspace{2.97in}\= \kill % set up two tab positions          
   \textbf{Tic-Tac-Toe}\\
   \texttt{github.com/tesa4436/tictactoe}\>\>Language: C\\
   \end{tabbing}\vspace{-20pt}
   A classic tic-tac-toe game in the server/client model, using Internet domain BSD sockets. 
   Features a custom protocol to facilitate the communication between the server and the client.

   \begin{tabbing}
   \hspace{2.3in}\= \hspace{2.8in}\= \kill % set up two tab positions          
   \textbf{PacMan}\\
   \texttt{github.com/tesa4436/pacman}\>\>Language: Java\\
   \end{tabbing}\vspace{-20pt}
   This is a PacMan game featuring random level generation (modified depth-first search) and
   E. Dijkstra's pathfinding algorithm with a minor alteration.

\section{SKILLS}          
\begin{itemize}
	\item Languages: C, C\#, Java, C++, SQL
	\item \LaTeX{} document typesetting
	\item Git, Vim, shell scripting, GNU debugger
	\item Linux system administration
\end{itemize}

\section{HOBBIES}
\begin{itemize}
	\item Cycling 	
	\item Hiking, mountain climbing
	\item Cross-country skiing 	
	\item Photography
\end{itemize}

\end{resume}
\end{document}
